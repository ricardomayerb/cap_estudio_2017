\documentclass[12pt,a4paper]{article}
\usepackage[utf8]{inputenc}
\usepackage{amsmath}
\usepackage{amsfonts}
\usepackage{amssymb}
\usepackage[spanish]{babel}
\usepackage[right=2cm,left=2cm,top=2cm,bottom=2cm,headsep=0cm,footskip=0.5cm]{geometry}
%\usepackage{makeidx}
\author{Ricardo Mayer}
\title{Nueva Taxonomía}
\begin{document}
\maketitle
\tableofcontents
	
\section{Introducción: reconociendo la heterogeneidad en la región}

\textbf{Breve descripción:} \newline

\begin{enumerate}
	\item \textbf{Explicar que propondremos una taxonomía} de países basada en los riesgos externos que enfrenta.n y en las condiciones que exhiben para hacerles frente
	\item \textbf{Argumentar que la región es suficientemente heterogénea} como para necesitar estas distinciones a la hora de redactar alertas o sugerencias de política. Mientras algunos países tienen problemas más graves de solvencia fiscal o cuentan con un sistema financiero más frágil o superficial, haciéndolos muy vulnerables a shocks de tasas de interés o de restricciones crediticias, otros países tienen como principal desafío cambiar la orientación del gasto doméstico hacia formas de inversión productiva que permitan la parte expansiva del los ciclos económicos sean más prolongadas y robustas, siendo su principal riesgo los shocks de demanda externa. Tiene sentido que los discursos de política económica traten estos países en forma diferenciada.	
	\item \textbf{Adelantar un ejemplo} donde, para un mismo año, las preocupaciones centrales de dos países eran muy distintas, por lo que políticas que apuntaran al promedio o que obviaran uno de los dos países, dejan mucho que desear.	
\end{enumerate}

\textbf{Muestra:} \newline

%El objetivo de este capítulo es proponer una taxonomía de países, pensada para América Latina y el Caribe, de acuerdo a los distintos riesgos externos que enfrentan y a las distintas condiciones que tienen para afrontar y superar los efectos adversos esos shocks.  
%
%Mientras que las recomendaciones de política, alertas y felicitaciones dirigidas a una suerte de agente representativo de la región o a un promedio de realidades, puede tener utilidad y en ocasiones representar a una buena mayoría de los países de LAC



\subsection{Mecanismos de propagación y mecanismos de reactivación}	

\textbf{Breve descripción:} \newline

\begin{enumerate}
	\item \textbf{Una recomendación de política debe considerar tanto los factores que hacen vulnerable a una economía y los que le permiten salir del problema con rapidez y solidez}. Ejercicios a lo \textit{stress test}, suelen fijarse más que nada en las vulnerabilidades y la magnitud y duración de la caída. Eso nos parece bien como primer criterio de diferenciación, pero creemos que es importante mirar además la posición que tiene el país para recuperarse después de estas caídas. No da lo mismo que sus recuperaciones sean frágiles o robustas, concentradas o diversificadas, breves o largas, etc.
\end{enumerate}


\textbf{Muestra:} \newline

\newpage	
\section{Indicadores de \textit{salud} y exposición a riesgos}
\subsection{Macro vulnerabilidades y determinantes de las recuperaciones}
\textbf{Breve descripción:} \newline

\begin{enumerate}
	\item Macro vulnerabilidades son aquellas variables que nos acercan a la probabilidad de caer en default o de iniciar una recesión (inflación no?) 
	\item Escribir una lista no exhaustiva de elementos que pertenezcan el primer sentido (vulnerabilidad y exposición) y otra al segundo (recuperación)
	\item Anunciar que a continuación discutiremos características importantes de la economía, que están ligadas a estas vulnerabilidades y potencialidades, dejando para la sección siguiente la confección de los indicadores. 	
\end{enumerate}



\textbf{Muestra:} \newline

Ejemplo tomado de \textit{Horizontes 2030}:

\textit{Las economía de América Latina y el caribe están expuestas fundamentalmente a dos tipos de choques externos: los reales, determinados por movimientos de los términos de intercambio o la variación del ritmo de crecimiento de los principales socios comerciales del país, y los financieros, asociados a las fluctuaciones de los flujos de inversión externa de corto y largo plazo. \newline \newline
La vulnerabilidad externa real depende de la especialización comercial de cada país. Un menor grado de diversificación productiva o un mayor grado de concentración exportadora en unos pocos socios comerciales expone excesivamente a una economía. La fuerte dependencia de varios países de Centroamérica ye l Caribe de las remesas provenientes del exterior o del turismo receptivo constituye el mismo tipo de vulnerabilidad. La vulnerabilidad externa financiera depende, por su parte, del grado de apalancamiento externo de cada economía, incluido el mayor o menor grado de penetración de la IED, lo que a su vez depende del grado de apertura financiera y del marco regulatorio de la radicación de capitales externos. Este tipo de vulnerabilidad se manifiesta en una posición patrimonial desfavorable, caracterizada por elevados ratios de endeudamiento. A mayor apalancamiento externo, mayor exposición a reversiones repentinas del ciclo financiero internacional (sudden stops) o a modificaciones de la política monetaria de los países centrales}


Pero falta un resumen similar para la potencialidad de recuperación.

\vspace{0.5cm}
\subsection{Política Monetaria}
\textbf{Breve descripción:} \newline

\begin{enumerate}
    \item Lo principal es evaluar con qué herramientas cuenta la autoridad monetaria, el espacio y la eficacia con que las puede emplear: 
    \item pensando en las vulnerabilidades  hay suficientes reservas para garantizar que las divisas no se disparen en precio en caso de una contracción violenta de la oferta de renminbi, dólares o euros? Tiene el país acceso a facilidades de liquidity swap con la Fed, el ECB o el Banco Popular Chino?
    \item pensando en la recuperación: Tiene espacio para bajar la tasa de política (está cerca del cero o la inflación viene subiendo o está muy fuera del rango)? Es esperable que las tasas comerciales se muevan junto a la de política monetaria?	Cuanto reaccionan las exportaciones netas al spread de tasas internacionales?
    \item Qué instrumentos estarían amparados por la figura del Lender of last resort? que proporción del sistema financiero representa?
    \item Como está la hoja de balance del banco central? puede hacerse cargo de activos financieros ilíquidos o muy caros para incentivar la compra de activos de largo plazo por parte del mercado?
\end{enumerate}

\subsubsection{Inflación y política de precios agregados}
\textbf{Breve descripción:} \newline

\begin{enumerate}
	\item Están dentro del rango objetivo? si está desviados, por cuanto tiempo han estado off-target? cuan amplio es el desvío?
	\item Cuál es la tendencia de los últimos períodos?
\end{enumerate}

\subsubsection{Reservas Internacionales y tipo de cambio}
\textbf{Breve descripción:} \newline

\begin{enumerate}
	\item  Si la economía no está dolarizada, hay espacio para que la moneda local se aprecie sin eliminar la rentabilidad de los exportadores? hay espacio para una depreciación de la moneda local sin eliminar la rentabilidad de los importadores?
	\item  Hay reservas suficientes para enfrentar un situación transitoria de menor liquidez de divisas por un par de trimestres?
\end{enumerate}

\subsubsection{Efectividad de la política y restricciones a los instrumentos}
\textbf{Breve descripción:} \newline

\begin{enumerate}
	\item Está la tasa de política cerca de la zero lower bound?
	\item Hay evidencia de cuan cerca el crédito y las tasas comerciales siguen a la tasa de política monetaria?	
\end{enumerate}



\vspace{0.5cm}
\subsection{Indicadores Fiscales}
\textbf{Breve descripción:} \newline

\begin{enumerate}
	\item Dos dimensiones importantes: solvencia-liquidez y espacio-eficacia del gasto
	\item La deuda de empresas estatales debe fluctuar bastante más que la del gobierno central y está fuertemente determinada por precios, en el caso de las empresas productivas. Cómo cambian las obligaciones del gobierno central si tuvieran que salir a capitalizar sus empresas? Cuánto cambia su liquidez y su grado de solvencia?
	\item Cuánto del gasto público es consumo y cuanto inversión? Tiene buenos canales para realizar inversión pública? En qué sectores invierte? Dirigida o across the board en un sector (e.g. en educación)? Hace el gobierno inversión pública contra cíclica?
	\item Como es la composición de la deuda pública? Hay un buen pareo de maturities y monedas entre activos y pasivos? 
	\item 
\end{enumerate}

\subsubsection{Consumo versus inversión}

\subsubsection{Balance del gobierno central}

\subsubsection{Deuda Soberana}

\subsubsection{Balance de empresas públicas}
%\paragraph{Nivel de deuda}
%\paragraph{Maduración}
%\paragraph{Divisas versus LCU}
%\paragraph{Exposición financiera}




\vspace{0.5cm}

\vspace{0.5cm}
\subsection{Sector externo}
\textbf{Breve descripción:} \newline

\begin{enumerate}
	\item Está demasiado concentrada la canasta exportadora?
	\item Está demasiado concentrada en uno o dos compradores?
	\item Tendencias en los últimos años en estas dos dimensiones
	\item Que proporción del GDP, del empleo y de los ingresos fiscales representa el sector exportador? Cuál es la exposición del sector público a una caída del 10 por ciento en el precio de las exportaciones? Distinguir entre exposición directa via recaudación y exposición indirecta vía aumento de subsidios o transferencia de recursos a SOEs exportadoras
\end{enumerate}



\vspace{0.5cm}
\subsection{Sector financiero}
\textbf{Breve descripción:} \newline

\begin{enumerate}
	\item La hoja de balance muestra si están concentrados en activos muy riesgosos? de muy corto plazo? Demasiado apalancados? El sector regulado es muy pequeño (cubre areas no tan grandes) como el sector sombrío? Cuan efectivo es el sistema financiero para llegar a eventuales cuentapropistas durante o justo despues de una crisis?
	\item Cuan capitalizado se encuentra el sector financiero privado?
	\item Cuan protegido se encuentra por un lender of last resort?
	\item Cuan concentrada está la industria (pensando en too big too fail) 
	\item Que proporción de sus activos se encuentran afuera? donde?
\end{enumerate}



\vspace{0.5cm}

\subsection{Inversión}
\textbf{Breve descripción:} \newline

Comentar las tendencias recientes de la inversión domestica y extranjera, pública y privada, con especial atención a los niveles que en el pasado permitieron tasas altas de crecimiento del GDP (20 por ciento?). Algo así como un investment gap de cada país respecto de sus mejores épocas. Poner atención si los ciclos se han vuelto más atenuados y si han copntibuído o no a moderar el efecto de los choques externos.

\begin{enumerate}
	\item Hay indicios de inversión contra cíclica, por ejemplo pública?
	\item Nivel y tendencia de la inversión en infraestructura
	\item IED, greenfield investment, brownfield investment e inversion financiera maquilladas
	\item Plan de concesiones
\end{enumerate}



\subsubsection{Inversión agregada y sectorial}


\subsubsection{Doméstica y extranjera directa}





\vspace{0.5cm}

\subsection{Empleo}

\textbf{Breve descripción:} \newline
Esta categoría debería esclarecer si el país está en un emergencia económica donde cualquier creación de empleo es bienvenida, por ejemplo alto desempleo y poco empleo por cuenta propia, o está en una etapa donde el desafío es tomar empleos precarios del sector cuentapropista y transformarlo en empleos formales con mayor estabilidad y acceso a protección social.

Un objetivo secundario, es establecer con cuanta dificultad trabajadores de un sector golpeado pueden emplearse en otro, o si definitivamente pasan al desempleo o el empleo por cuenta propia por un período largo.

%\begin{enumerate}
%	\item 
%	\item 
%	\item 
%	\item 
%\end{enumerate}

\subsubsection{Composición: cuenta propia, empleados, estacionales. Agregado y por sectores}

\subsubsection{Elasticidad empleo producto, agregado y por sectores}

\subsubsection{Hay reempleo, reconversión, de un sector económico a otro?}

\subsection{Otras características estructurales}

\subsubsection{Concentración y competencia por sector}

\subsubsection{Competitividad internacional sectorial}



\vspace{0.5cm}
\subsection{Hogares}

\textbf{Breve descripción:} \newline
Lo más importante en esta sección es determinar si el consumo de los hogares puede ser un driver del crecimiento del PIB o no en caso de afrontar un choque externo negativo. O, en el otro extremo, si los niveles de morosidad son tan altos que amenazan la solvencia de algunas instituciones financieras. Para esto es necesario evaluar la carga financiera que actualmente soportan y la tendencia de la misma, junto con analizar las proyecciones de ingreso (que en parte se desprende del módulo de Empleo de esta misma sección). Para esta evaluación , deberíamos mirar antecedentes de encuestas financieras de hogares, junto con datos comerciales de morosidad o delinquency rates, deuda bruta reportada en instancias oficiales, etc. Finalmente, alguna referencia a activos financieros y físicos (por ejemplo, bienes inmuebles) debiera cerrar la apreciación de la hoja de balance de los hogares.

%\begin{enumerate}
%	\item 
%	\item 
%	\item 
%	\item 
%\end{enumerate}

\subsubsection{Deudas y carga financiera}
\subsubsection{Participación laboral}
\subsubsection{Activos físicos}



\newpage
\section{Una nueva taxonomía con miras a los riesgos externos}

\textbf{Breve descripción:} \newline
En esta sección presentamos la clasificación de países, me imagino que tres o cuatro grupos. En cada grupo existen distintas exposiciones a choques externos y distintas fortalezas o debilidad para acometer la recuperación post choques. Aquí se describe el perfil resumido de cada grupo.


\vspace{0.5cm}
Por ejemplo: 

\textit{\textbf{Las características del Grupo 1} implican alto riesgo de insolvencia fiscal de la que se saldrá lentamente y a un costo alto o de una contracción severa y prolongada del producto y del empleo. En estos casos la prioridad debe estar en recuperar la hoja de balance del gobierno, ya sea por una combinación de financiamiento y reestructuración de la deuda o por cambios que razonablemente aseguren superavit primarios a mediano y largo plazo. \newline
\textbf{El Grupo 4, en tanto}, tiene una posición fiscal más holgada y aunque su canasta de exportaciones no está suficientemente diversificada, cuenta con el espacio de política monetaria y fiscal suficiente para moderar la parte negativa del ciclo. El énfasis de política debería estar puesto en desarrollar políticas que estimulen la inversión pública o privada para producir recuperaciones más vigorosas y duraderas. Por ejemplo en el caso de Chile, sería importante que \ldots
}


\vspace{0.5cm}
Finalmente, en un \textbf{apéndice} del capítulo debemos describir la metodología usada en la confección de los grupos e idealmente reportar que hicimos algunas pruebas de robustez y que los grupos nos daban bastante parecidos aún cuando cambiáramos algunos criterios e indicadores. Más detalles de esos ejercicios pueden quedar un apéndice on-line, en la web.
\vspace{0.5cm}
  \subsection{Grupos sugeridos, perfil y miembros en el 2017}
  \subsection{Diferentes exposiciones al riesgo across groups}
  \subsection{Diferentes potenciales de recuperación across groups}
  \subsection{Implicancias de política diferentes across groups}
 


\newpage
\section{Ejemplos de recomendaciones diferenciadas por grupo}

Podemos tomar dos ejemplos de recomendaciones clásicas, ya sea sugeridas por Cepal alguna vez o por el IMF o el WB para la región: recorte de gasto público, mejorar la eficiencia de la recaudación, aumentar reservas internacionales, dejar flotar el tipo de cambio \ldots y mostrar a la luz de nuestra clasificación como parecen muy pertinentes para un grupo y bastante irrelevantes o perjudiciales para otro, alertando entonces del peligro de hacer recomendaciones a la bandada completa.

Por ejemplo: En 1998, con el comienzo de la crisis asiática, la recomendación de política del IMF para la región consistió en \ldots en circuntacias que miebntras que era una recomendación enteramente pertinente para los países x,y,z que se encontraban en lo que ahora llamaríamos Grupo 2, tenía poco sentido como tarea para los países a,b,c que en ese entonces tenían desafíos más propios de lo que hemos identificado como Grupo 1, y que por lo tanto, debían estar más preocupados de hacer \ldots

\section{Conclusiones}







	
\end{document}